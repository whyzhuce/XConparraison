%%%%%%%%%%%%%%%%%%%%%%%%%%%%%%%%%%%%%%%%%
% Journal Article
% LaTeX Template
% Version 1.4 (15/5/16)
%
% This template has been downloaded from:
% http://www.LaTeXTemplates.com
%
% Original author:
% Frits Wenneker (http://www.howtotex.com) with extensive modifications by
% Vel (vel@LaTeXTemplates.com)
%
% License:
% CC BY-NC-SA 3.0 (http://creativecommons.org/licenses/by-nc-sa/3.0/)
%
%%%%%%%%%%%%%%%%%%%%%%%%%%%%%%%%%%%%%%%%%

%----------------------------------------------------------------------------------------
%	PACKAGES AND OTHER DOCUMENT CONFIGURATIONS
%----------------------------------------------------------------------------------------

\documentclass[twoside,twocolumn]{article}

\usepackage{blindtext} % Package to generate dummy text throughout this template 

\usepackage[sc]{mathpazo} % Use the Palatino font
\usepackage[T1]{fontenc} % Use 8-bit encoding that has 256 glyphs
\linespread{1.05} % Line spacing - Palatino needs more space between lines
\usepackage{microtype} % Slightly tweak font spacing for aesthetics

\usepackage[english]{babel} % Language hyphenation and typographical rules

\usepackage[hmarginratio=1:1,top=32mm,columnsep=20pt]{geometry} % Document margins
\usepackage[hang, small,labelfont=bf,up,textfont=it,up]{caption} % Custom captions under/above floats in tables or figures
\usepackage{booktabs} % Horizontal rules in tables

\usepackage{lettrine} % The lettrine is the first enlarged letter at the beginning of the text

\usepackage{enumitem} % Customized lists
\setlist[itemize]{noitemsep} % Make itemize lists more compact

\usepackage{abstract} % Allows abstract customization
\renewcommand{\abstractnamefont}{\normalfont\bfseries} % Set the "Abstract" text to bold
\renewcommand{\abstracttextfont}{\normalfont\small\itshape} % Set the abstract itself to small italic text

\usepackage{titlesec} % Allows customization of titles
\renewcommand\thesection{\Roman{section}} % Roman numerals for the sections
\renewcommand\thesubsection{\roman{subsection}} % roman numerals for subsections
\titleformat{\section}[block]{\large\scshape\centering}{\thesection.}{1em}{} % Change the look of the section titles
\titleformat{\subsection}[block]{\large}{\thesubsection.}{1em}{} % Change the look of the section titles

\usepackage{fancyhdr} % Headers and footers
\pagestyle{fancy} % All pages have headers and footers
\fancyhead{} % Blank out the default header
\fancyfoot{} % Blank out the default footer
\fancyhead[C]{Running title $\bullet$ May 2016 $\bullet$ Vol. XXI, No. 1} % Custom header text
\fancyfoot[RO,LE]{\thepage} % Custom footer text

\usepackage{titling} % Customizing the title section

\usepackage{hyperref} % For hyperlinks in the PDF

\usepackage{amsmath}
\usepackage{amssymb}
%----------------------------------------------------------------------------------------
%	TITLE SECTION
%----------------------------------------------------------------------------------------

\setlength{\droptitle}{-4\baselineskip} % Move the title up

\pretitle{\begin{center}\Huge\bfseries} % Article title formatting
\posttitle{\end{center}} % Article title closing formatting
\title{Pricing with imperfect collateral} % Article title
\author{%
\textsc{Chen Xu} \\%\thanks{A thank you or further information} \\[1ex] % Your name
\normalsize Barclays Singapore, QA Macro Rates \\ % Your institution
\normalsize %\href{mailto:john@smith.com}{john@smith.com} % Your email address
%\and % Uncomment if 2 authors are required, duplicate these 4 lines if more
%\textsc{Jane Smith}\thanks{Corresponding author} \\[1ex] % Second author's name
%\normalsize University of Utah \\ % Second author's institution
%\normalsize \href{mailto:jane@smith.com}{jane@smith.com} % Second author's email address
}
\date{\today} % Leave empty to omit a date
\renewcommand{\maketitlehookd}{%
\begin{abstract}
\noindent Swaps cleared at LCH are daily margined under OIS rate. However OIS markets don't exist in certain currency markets. Hence PV of such swaps at LCH use different discounting curves (e.g. SOR6M for SGD). This note explores the valuation impact under Black-Scholes of using discounting curve other than the curve of the corresponding collateral remuneration rate.  %\blindtext % Dummy abstract text - replace \blindtext with your abstract text
\end{abstract}
}

%----------------------------------------------------------------------------------------

\begin{document}

% Print the title
\maketitle

%----------------------------------------------------------------------------------------
%	ARTICLE CONTENTS
%----------------------------------------------------------------------------------------

\section{Introduction}

\lettrine[nindent=0em,lines=3]{P} iterbarg\cite{Piterbarg:2010} discussed derivative pricing under fully collateralized and uncollateralized case. Singapore Dollar(SGD) swaps cleared at LCH are remunerated at OIS rate. However OIS derivative market doesn't exist for SGD (there's no OIS curve), Mark-to-Market at LCH of these swaps are discounted using SOR6M curve. This disconnection of margin calculation versus the actual collateral remuneration operation turns out to have secondary effect on the valuation. 

%\blindtext % Dummy text

%\blindtext % Dummy text

%------------------------------------------------

\section{Black-Scholes with collateral}

Using Piterbarg\cite{Piterbarg:2010}'s notation. Let $S(t)$ be an asset that follows, in the real world, the following dynamics:
\begin{equation}
\frac{dS(t)}{S(t)}= \mu_{S}(t)dt+\sigma_{S}(t)dW(t) \nonumber
\end{equation}

Let $V(t, S)$ be a derivatives security on the asset; by replication and Ito's lemma we have:
\begin{align*}
&\frac{\partial V}{\partial t} + (r_{R}(t)-r_{D}(t))\frac{\partial V}{\partial S}S + \frac{\sigma_{S}(t)^2}{2}S^2\frac{\partial^2 V}{\partial S^2}\\
&= r_{F}(t)V(t) - (r_{F}(t) - r_{C}(t))C(t)
\end{align*}

where $r_{R}(t)$ is the short rate on funding secured by the asset ('repo'); $r_{F}(t)$ is the unsecured funding rate; $r_{C}(t)$ is the overnight rate paid on the collateral ('CSA') and $r_{D}(t)$ is the dividend rate if the underlying is stock.


Using Feynman-Kac, we have the solution:
\begin{align*}
&V(t) = E_t [ e^{-\int_t^T(r_{F}(u) du)}V(T) \\
& + \int_t^T e^{-\int_t^T r_{F}(v)dv} (r_{F}(u) - r_{C}(u)) C(u) du ]
\end{align*}
in the measure in which the underlying asset grows at rate $r_{R}(t)-r_{D}(t)$.


$\blacksquare$ When $C(t) = 0$, the value of the derivative:
\begin{align*}
V(t) = E_t [ e^{-\int_t^T(r_{F}(u) du)}V(T)]
\end{align*}

$\blacksquare$ When $C(t) = V(t)$, the value of the derivative:
\begin{align*}
V(t) = E_t [ e^{-\int_t^T(r_{C}(u) du)}V(T)]
\end{align*}

$\blacksquare$ Let's consider the collateral is valued using a different discounting curve $r_{M}(t)$, i.e.:
\begin{align*}
C(t) = E_t [ e^{-\int_t^T(r_{M}(u) du)}V(T)]
\end{align*}

Hence the valuation PDE becomes:
\begin{align*}
&dV(t)=r_{F}(t)V(t)dt - (r_{F}(t)-r_{C}(t))C(t)dt \\
&dC(t)=r_{M}C(t)dt
\end{align*}

Define $\alpha(t)$ and $Z(t) = V(t)+\alpha(t)C(t)$ such that:
\begin{align*}
dZ(t) = r_{F}(t) Z(t) dt
\end{align*}
This means 
\begin{align*}
&\frac{dZ(t)}{dt} = r_{F}(t)V(t) - (r_{F}(t)-r_{C}(t))C(t) \\
&+ \alpha(t)r_{M}(t)C(t) +  \frac{d\alpha(t)}{dt}C(t) \\
&= r_{F}(t)(V(t)+\alpha(t)C(t))
\end{align*}
We have:
\begin{align*}
\frac{d\alpha(t)}{dt} + (r_{M}(t)-r_{F}(t)) \alpha(t) + r_{C}(t)-r_{F}(t) = 0
\end{align*}

We can choose:
\begin{align*}
\alpha(s) = \frac{r_{F}(s)-r_{C}(s)}{r_{M}(s)-r_{F}(s)}(1-e^{-\int_t^s (r_{M}(u)-r_{F}(u))du})
\end{align*}
Hence:
\begin{align*}
V(t)& = E_t[ e^{-\int_t^T(r_{F}(u) du)}V(T)(1+\alpha(T))] \\
&- \alpha(t)E_t [ e^{-\int_t^T(r_{M}(u) du)}V(T)] \\
=& E_t[V(T) ( \frac{r_{F}(s)-r_{C}(s)}{r_{F}(s)-r_{M}(s)} e^{-\int_t^T(r_{M}(u) du} \\
&- \frac{r_{M}(s)-r_{C}(s)}{r_{F}(s)-r_{M}(s)} e^{-\int_t^T(r_{F}(u) du})]
\end{align*}

% \begin{itemize}
% \item Donec dolor arcu, rutrum id molestie in, viverra sed diam
% \item Curabitur feugiat
% \item turpis sed auctor facilisis
% \item arcu eros accumsan lorem, at posuere mi diam sit amet tortor
% \item Fusce fermentum, mi sit amet euismod rutrum
% \item sem lorem molestie diam, iaculis aliquet sapien tortor non nisi
% \item Pellentesque bibendum pretium aliquet
% \end{itemize}
% \blindtext % Dummy text

% Text requiring further explanation\footnote{Example footnote}.

%------------------------------------------------

\section{Results}

% \begin{table}
% \caption{Example table}
% \centering
% \begin{tabular}{llr}
% \toprule
% \multicolumn{2}{c}{Name} \\
% \cmidrule(r){1-2}
% First name & Last Name & Grade \\
% \midrule
% John & Doe & $7.5$ \\
% Richard & Miles & $2$ \\
% \bottomrule
% \end{tabular}
% \end{table}

\blindtext % Dummy text

\begin{equation}
\label{eq:emc}
e = mc^2
\end{equation}

\blindtext % Dummy text

%------------------------------------------------

\section{Discussion}

\subsection{Subsection One}

A statement requiring citation %\cite{Figueredo:2009dg}.
\blindtext % Dummy text

\subsection{Subsection Two}

\blindtext % Dummy text

%----------------------------------------------------------------------------------------
%	REFERENCE LIST
%----------------------------------------------------------------------------------------

\begin{thebibliography}{99} % Bibliography - this is intentionally simple in this template

\bibitem{Piterbarg:2010}
Piterbarg, V. (2010).
\newblock Funding beyond discounting: collateral agreements and derivatives pricing.\\
\newblock {\em Cutting edge, AsiaRisk, March}
 
\end{thebibliography}

%----------------------------------------------------------------------------------------

\end{document}
