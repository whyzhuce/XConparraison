%%%%%%%%%%%%%%%%%%%%%%%%%%%%%%%%%%%%%%%%%
% Journal Article
% LaTeX Template
% Version 1.4 (15/5/16)
%
% This template has been downloaded from:
% http://www.LaTeXTemplates.com
%
% Original author:
% Frits Wenneker (http://www.howtotex.com) with extensive modifications by
% Vel (vel@LaTeXTemplates.com)
%
% License:
% CC BY-NC-SA 3.0 (http://creativecommons.org/licenses/by-nc-sa/3.0/)
%
%%%%%%%%%%%%%%%%%%%%%%%%%%%%%%%%%%%%%%%%%

%----------------------------------------------------------------------------------------
%	PACKAGES AND OTHER DOCUMENT CONFIGURATIONS
%----------------------------------------------------------------------------------------

\documentclass[twoside,twocolumn]{article}

\usepackage{blindtext} % Package to generate dummy text throughout this template 

\usepackage[sc]{mathpazo} % Use the Palatino font
\usepackage[T1]{fontenc} % Use 8-bit encoding that has 256 glyphs
\linespread{1.05} % Line spacing - Palatino needs more space between lines
\usepackage{microtype} % Slightly tweak font spacing for aesthetics

\usepackage[english]{babel} % Language hyphenation and typographical rules

\usepackage[hmarginratio=1:1,top=32mm,columnsep=20pt]{geometry} % Document margins
\usepackage[hang, small,labelfont=bf,up,textfont=it,up]{caption} % Custom captions under/above floats in tables or figures
\usepackage{booktabs} % Horizontal rules in tables

\usepackage{lettrine} % The lettrine is the first enlarged letter at the beginning of the text

\usepackage{enumitem} % Customized lists
\setlist[itemize]{noitemsep} % Make itemize lists more compact

\usepackage{abstract} % Allows abstract customization
\renewcommand{\abstractnamefont}{\normalfont\bfseries} % Set the "Abstract" text to bold
\renewcommand{\abstracttextfont}{\normalfont\small\itshape} % Set the abstract itself to small italic text

\usepackage{titlesec} % Allows customization of titles
\renewcommand\thesection{\Roman{section}} % Roman numerals for the sections
\renewcommand\thesubsection{\roman{subsection}} % roman numerals for subsections
\titleformat{\section}[block]{\large\scshape\centering}{\thesection.}{1em}{} % Change the look of the section titles
\titleformat{\subsection}[block]{\large}{\thesubsection.}{1em}{} % Change the look of the section titles

\usepackage{fancyhdr} % Headers and footers
\pagestyle{fancy} % All pages have headers and footers
\fancyhead{} % Blank out the default header
\fancyfoot{} % Blank out the default footer
\fancyhead[C]{Running title $\bullet$ May 2016 $\bullet$ Vol. XXI, No. 1} % Custom header text
\fancyfoot[RO,LE]{\thepage} % Custom footer text

\usepackage{titling} % Customizing the title section

\usepackage{hyperref} % For hyperlinks in the PDF

\usepackage{amsmath}
\usepackage{amssymb}
\usepackage{graphicx}
%----------------------------------------------------------------------------------------
%	TITLE SECTION
%----------------------------------------------------------------------------------------

\setlength{\droptitle}{-4\baselineskip} % Move the title up

\pretitle{\begin{center}\Huge\bfseries} % Article title formatting
\posttitle{\end{center}} % Article title closing formatting
\title{Pricing with imperfect collateral} % Article title
\author{%
\textsc{Chen Xu} \\%\thanks{A thank you or further information} \\[1ex] % Your name
% \normalsize Barclays Singapore, QA Macro Rates \\ % Your institution
\normalsize %\href{mailto:john@smith.com}{john@smith.com} % Your email address
%\and % Uncomment if 2 authors are required, duplicate these 4 lines if more
%\textsc{Jane Smith}\thanks{Corresponding author} \\[1ex] % Second author's name
%\normalsize University of Utah \\ % Second author's institution
%\normalsize \href{mailto:jane@smith.com}{jane@smith.com} % Second author's email address
}
\date{\today} % Leave empty to omit a date
\renewcommand{\maketitlehookd}{%
\begin{abstract}
\noindent Swaps cleared at LCH are daily margined under OIS rate. However OIS markets don't exist in certain currency markets. Hence PV of such swaps at LCH use different discounting curves (e.g. SOR6M for SGD). This note shows that the valuation impact of using discounting curve other than the curve of the corresponding collateral remuneration rate is relatively small under Black-Scholes.  %\blindtext % Dummy abstract text - replace \blindtext with your abstract text
\end{abstract}
}

%----------------------------------------------------------------------------------------

\begin{document}

% Print the title
\maketitle

%----------------------------------------------------------------------------------------
%	ARTICLE CONTENTS
%----------------------------------------------------------------------------------------

\section{Introduction}

\lettrine[nindent=0em,lines=3]{D}erivative pricing under fully collateralized and uncollateralized case are discussed in Piterbarg\cite{Piterbarg:2010}. It is known that the collateral remuneration rate is the equivalent risk free rate used in the valuation as well as daily mark-to-market discounting. For example, most swaps cleared at LCH are remunerated at OIS rate and OIS curve is used in daily PV calculation. However OIS derivative market doesn't exist for SGD (there's no OIS curve), mark-to-market at LCH of these swaps are discounted using SOR6M curve. These is a clear disconnection of margin discounting calculation versus the actual collateral remuneration operation. 

%\blindtext % Dummy text

%\blindtext % Dummy text

%------------------------------------------------

\section{Black-Scholes with collateral}

Using Piterbarg\cite{Piterbarg:2010}'s notation. Let $S(t)$ be an asset that follows, in the real world, the following dynamics:
\begin{equation}
\frac{dS(t)}{S(t)}= \mu_{S}(t)dt+\sigma_{S}(t)dW(t) \nonumber
\end{equation}

Let $V(t, S)$ be a derivatives security on the asset; by replication and Ito's lemma we have:
\begin{align*}
&\frac{\partial V}{\partial t} + (r_{R}(t)-r_{D}(t))\frac{\partial V}{\partial S}S + \frac{\sigma_{S}(t)^2}{2}S^2\frac{\partial^2 V}{\partial S^2}\\
&= r_{F}(t)V(t) - (r_{F}(t) - r_{C}(t))C(t)
\end{align*}

where $r_{R}(t)$ is the short rate on funding secured by the asset ('repo'); $r_{F}(t)$ is the unsecured funding rate; $r_{C}(t)$ is the overnight rate paid on the collateral ('CSA') and $r_{D}(t)$ is the dividend rate if the underlying is stock.


Using Feynman-Kac, we have the solution:
\begin{align*}
&V(t) = E_t [ e^{-\int_t^T(r_{F}(u) du)}V(T) \\
& + \int_t^T e^{-\int_t^T r_{F}(v)dv} (r_{F}(u) - r_{C}(u)) C(u) du ]
\end{align*}
in the measure in which the underlying asset grows at rate $r_{R}(t)-r_{D}(t)$.


$\blacksquare$ When $C(t) = 0$, the value of the derivative:
\begin{align*}
V(t) = V_{F}(t) := E_t [ e^{-\int_t^T(r_{F}(u) du)}V(T)]
\end{align*}

$\blacksquare$ When $C(t) = V(t)$, the value of the derivative:
\begin{align*}
V(t) = V_{C}(t) := E_t [ e^{-\int_t^T(r_{C}(u) du)}V(T)]
\end{align*}

$\blacksquare$ Let's consider the case where collaterals are valued under a different discounting curve $r_{M}(t)$, i.e.:
\begin{align*}
C(t) = V_{M}(t) := E_t [ e^{-\int_t^T(r_{M}(u) du)}V(T)]
\end{align*}

Hence the valuation PDE becomes:
\begin{align*}
&dV(t)=r_{F}(t)V(t)dt - (r_{F}(t)-r_{C}(t))C(t)dt \\
&dC(t)=r_{M}C(t)dt
\end{align*}

Define $\alpha(t)$ and $Z(t) = V(t)+\alpha(t)C(t)$ such that:
\begin{align*}
dZ(t) = r_{F}(t) Z(t) dt
\end{align*}
Applying Ito, this new portfolio follows:
\begin{align*}
&\frac{dZ(t)}{dt} = r_{F}(t)V(t) - (r_{F}(t)-r_{C}(t))C(t) \\
&+ \alpha(t)r_{M}(t)C(t) +  \frac{d\alpha(t)}{dt}C(t) \\
&= r_{F}(t)(V(t)+\alpha(t)C(t))
\end{align*}
We obtain the condition of $\alpha(t)$ for such portfolio to exist:
\begin{align*}
\frac{d\alpha(t)}{dt} - (r_{F}(t)-r_{M}(t)) \alpha(t) - r_{F}(t)-r_{C}(t)=0
\end{align*}

We can choose so that $\alpha(T)=0$:
\begin{align*}
\alpha(t)=\int_T^t e^{\int_s^t (r_{F}(u)-r_{M}(u))du} (r_{F}(s)-r_{C}(s))ds
% \alpha(s) = \frac{r_{F}(s)-r_{C}(s)}{r_{M}(s)-r_{F}(s)}(1-e^{-\int_t^s (r_{M}(u)-r_{F}(u))du})
\end{align*}

Hence the PV of the derivative contract becomes:
\begin{align*}
V(t)& = E_t[ e^{-\int_t^T(r_{F}(u) du)}V(T)(1+\alpha(T))] \\
&- \alpha(t)E_t [ e^{-\int_t^T(r_{M}(u) du)}V(T)] \\
% =& E_t[V(T) ( \frac{r_{F}(s)-r_{C}(s)}{r_{F}(s)-r_{M}(s)} e^{-\int_t^T(r_{M}(u) du} \\
% &- \frac{r_{M}(s)-r_{C}(s)}{r_{F}(s)-r_{M}(s)} e^{-\int_t^T(r_{F}(u) du})]
&=V_{F}(t)-\alpha(t) V_{M}(t)
\end{align*}

% \begin{itemize}
% \item Donec dolor arcu, rutrum id molestie in, viverra sed diam
% \item Curabitur feugiat
% \item turpis sed auctor facilisis
% \item arcu eros accumsan lorem, at posuere mi diam sit amet tortor
% \item Fusce fermentum, mi sit amet euismod rutrum
% \item sem lorem molestie diam, iaculis aliquet sapien tortor non nisi
% \item Pellentesque bibendum pretium aliquet
% \end{itemize}
% \blindtext % Dummy text

% Text requiring further explanation\footnote{Example footnote}.

%------------------------------------------------

% \section{Results}

% \begin{table}
% \caption{Example table}
% \centering
% \begin{tabular}{llr}
% \toprule
% \multicolumn{2}{c}{Name} \\
% \cmidrule(r){1-2}
% First name & Last Name & Grade \\
% \midrule
% John & Doe & $7.5$ \\
% Richard & Miles & $2$ \\
% \bottomrule
% \end{tabular}
% \end{table}

% \blindtext % Dummy text

% \begin{equation}
% \label{eq:emc}
% e = mc^2
% \end{equation}

% \blindtext % Dummy text

%------------------------------------------------

\section{Discussion}

% \subsection{Subsection One}

% A statement requiring citation %\cite{Figueredo:2009dg}.
% \blindtext % Dummy text

% \subsection{Subsection Two}

% \blindtext % Dummy text

$\blacksquare$ When $r_{M}(t) \xrightarrow{\forall t} \infty$, $\alpha(t) \xrightarrow{\forall t} 0$; hence $V(t) \xrightarrow{\forall t} V_{F}(t)$. This means that the mark-to-market of the collateral is always 0. This is equivalent of saying that the trade is actually uncollateralized: $V(t) = V_{F}(t)$. \\
$\blacksquare$ When $r_{M} = r_{C}$, $\alpha(t) = e^{\int_t^T -(r_{F}(s)-r_{C}(s))ds}-1$; hence $V(t) = V_{M}(t) = V_{C}(t)$. This means that the derivative is fully collateralized and its collateral PV is properly discounted at the remuneration rate: $V(t) = V_{C}(t)$. \\
$\blacksquare$ When $r_{M} = r_{F}$, $\alpha(t) = \int_T^t (r_{F}(s)-r_{C}(s))ds$; hence $V(t) = V_{F}(t) (1+\int_t^T (r_{F}(s)-r_{C}(s))ds)$.\\

$\blacksquare$ Assuming $r_{i}(t)$ is constant through time, we have $\alpha(t)=\frac{r_{F}-r_{C}}{r_{F}-r_{M}}(e^{-(r_{F}-r_{M})(T-t)}-1)$ when $r_{F} \neq r_{M}$.


$\square$ When $r_{M}$ in neighbourhood of $r_{F}$: $r_{M}=r_{F}+\delta r$. So $\alpha(t)=\frac{r_{F}-r_{C}}{\delta r}(e^{-\delta r(T-t)}-1)$. We have:
\begin{align*}
V(t) &= V_{F}(t) + V_{M}(t) (r_{F}-r_{C})\frac{1-e^{-\delta r(T-t)}}{\delta r}\\
&= V_{F} (1+e^{-\delta r(T-t)}(r_{F}-r_{C})\frac{1-e^{-\delta r(T-t)}}{\delta r})\\
&= V_{F} (1+(r_{F}-r_{C})(T-t)+\Theta(\delta))\\
&= V_{C}\frac{(1+(r_{F}-r_{C})(T-t)+\Theta(\delta))}{e^{(r_{F}-r_{C})(T-t)}}
\end{align*}
When $p=(r_{F}-r_{C})(T-t) \approx 1$ (say Libor3M-OIS basis is around 40bps and maturity is 5Y, $p=0.02 \ll 1$). We have:
$$V(t) = V_{C} (1+ \Theta(p+\delta r)).$$
This implies that $V(t)$ is pretty much dominated, in zero-th order, by the collateral remuneration rate $r_{C}$; while the MtM discounting rate $r_{M}$ has a first oder (even second order given $\delta r \ll p$) to the valuation of the derivative contract.

$\square$ Numerical illustration \ref{fig:Vm}: $r_{C}=2\%, r_{F}=2.5\%, t=0, T=5, V(T)=100.$
\begin{figure}
  \includegraphics[width=\linewidth]{figure_1.png}
  \caption{$V(t)$ in function of $r_{M}$}
  \label{fig:Vm}
 \end{figure}


%----------------------------------------------------------------------------------------
%	REFERENCE LIST
%----------------------------------------------------------------------------------------

\begin{thebibliography}{99} % Bibliography - this is intentionally simple in this template

\bibitem{Piterbarg:2010}
Piterbarg, V. (2010).
\newblock Funding beyond discounting: collateral agreements and derivatives pricing.\\
\newblock {\em Cutting edge, AsiaRisk, March 2010}
 
\end{thebibliography}

%----------------------------------------------------------------------------------------

\end{document}
